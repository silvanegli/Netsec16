\documentclass[12pt,a4paper]{article}
\usepackage[utf8]{inputenc}
\usepackage{hyperref}
\usepackage{graphicx}
\usepackage{subcaption}
\usepackage{float}
\usepackage{paralist}

% ToDo's
\usepackage{color}

\newcommand{\mynote}[3]{
  \textcolor{#2}{
    \fbox{\bfseries\sffamily\scriptsize#1}
    {\small\textsf{\emph{#3}}}
}}

\title{JWT and Bash Injection} %TODO or something more cryptic?
\author{Network Security HS16 - Group 6\\Lukas Bischofberger, Silvan Egli, Jonas Passerini, Lukas Widmer}

\begin{document}

\maketitle

\begin{abstract}
%a short summary of the challenge. You should include the problem or vulnerability addressed by the challenge and the approach used to exploit the vulnerability.

The basis of our challenge is a chat service. There are normal users and administrators. To authenticate the users JSON Web Tokens (JWT) are used. The administrators can make changes to the appearance of the website. The challenge consists of two consecutive tasks, where the ultimate goal is to extract a private key on the server. \\
The first tasks is to take advantage of a possible misconfiguration of JWT. We take advantage of the fact that certain implementations of JWT libraries allow the client to choose the signing algorithm.% The problem of JWT is that they need an algorithm for the verification step however an attacker can bypass this and so be seen as administrator. This is done in the way that the attacker forges his own token.\\
The second task is to exploit the system by use of a bash injection which is only possible with use of the administrator rights.% The changes of the appearance is done over the bash so a bash injection is possible. 
This vulnerability can then be used to extract the private key of the server.
\end{abstract}

\section{Challenge Description}

\subsection{Type of Challenge}
%whether the challenge is online or offline. For example, exploiting a web application requires the application to be online, whereas analyzing captured packets can be done offline.
The first task requires some offline work as one has to forge a token to get administrator rights. The rest of the challenge is an online as our chat service is a web application and the extraction of the private key will only be possible when the application is online.

\subsection{Category}
%one or more categories for the challenge. You can see a full list of possible challenges by going to the Hacking Lab home page, clicking the Statistics tab near the center of the page, and then clicking Challenge Details under the Challenge Statistics box.

The challenge belongs to the category web security. %TODO maybe also programming
\subsection{Mission}
%a summary of the challenge objective. For example, a school's online registration form may have an SQL injection vulnerability. Using SQL injection in this form to execute a DROP statement can cause a school to lose their entire year's student records.

There is web application (the chat service) hosted by some major company. Your task is it to break into this application as a weak entry point and steal the private key of the web application which you know is also used by other web applications hosted by this company. The objective can be achieved by using two vulnerabilities in the web application, first the JWT signing algorithm and second a bash injection.

%The objective of the first task is to become an administrator by using the JWT vulnerability. The vulnerability is that it is possible to sign the tokens oneself and send them with the 'none' algorithm.\\
%The objective of the second task is to exploit the bash injection vulnerability. There one has then to use this to extract the private key of the server by writing it to the html-file.
\subsection{Learning Goal}
%what you hope students will learn from the challenge and why it is important. For example, if the challenge is to perform a TLS-based man-in-the-middle attack using a similar-looking domain name, students will learn the importance of checking that the domain name in the certificate matches the one in the address bar. This is important because it is usually much easier to obtain a fraudulent certificate for a similar-looking domain name than for the actual domain.

The first task should teach the student about JWT and token based authentication in general. But also increase the awareness of the problem when the client can choose the signing algorithm. %This is especially a problem if it is also possible to not use any  algorithm. This has to be handled, if it is not done by the library itself.
The second task should make students aware of how user input needs to be sanitized when used as input to SQL or bash. %If it is really needed that users can make inputs to the bash, one should check the string for unintended commands or only allow certain strings to be executed.

\section{Implementation}
\subsection{Requirements}
%TODO hope this is ok like this, was not quite sure what to put here
We have to make a web interface, where users first have to get a JWT to determine if they are standard users or administrators. In the web interface itself the users can chat. The administrators can also apply changes to the appearance over special commands, which are written in the normal textbox like with Slack. These commands are then bash commands, which can change the html-file where the css-code is integrated.
\subsection{Deployment}
% Not quite sure what I should write here
We separated the work in several parts. On one hand there has to be Django, which uses JWT. Then also the background of the server.

\section{Solution}

\subsection{Hints}
%up to three hints to give to users solving the challenge.
We will give some hints like that we use JWT. This should help to find out the problem with the 'none' algorithm. Then in the chat history some special commands will be written, so that a user notices that an administrator can do more then just chat.
\subsection{Step-by-Step Instructions}
%a description of every step a user has to perform to solve the challenge. This does not need to be the final version, but it should allow us to get a feel for the challenge.
The first step is to forge a token. Then he has to send this token so that he becomes an administrator. Then he uses his new rights to write a bash command to write the private key to html file.
\section{Mitigation}
%how can the vulnerability be fixed or mitigated so that the challenge exploit does not work? For example, if the challenge vulnerability is that an HTTPS request can be intercepted and rewritten as an HTTP request, a mitigation is to have the server's response enforce the future use of HTTPS. This would protect against future attacks, though not an ongoing attack.

To tackle the first task one has to prevent that the 'none' algorithm can be used. However this can still lead to some problems so the best thing is to only accept one algorithm predefined by the server.\\
The problem of bash injection in the second task can be solved by either not using the text field for the changes one would like to make but buttons for example. An other way would be to check the strings for malicious substrings or even only allow certain strings and if a string does not match the predefined ones it is not accepted.

\section{Conclusion}
We made a challenge consisting of two general problems. On one hand not perfectly implemented access measures, which can lead to a permission increase by malicious users. On the other hand bash injections, which are the consequence of not properly checking inputs. Both are common problems, which could mostly be tackled in a simple way.
\end{document}
